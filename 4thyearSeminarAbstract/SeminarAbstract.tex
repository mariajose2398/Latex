\documentclass[12pt]{article}
\usepackage{graphicx, amsmath} % Required for inserting images
\usepackage[left=1in,top=1in,right=1in,bottom=1in]{geometry}
\title{\textbf{The Search for Self-Interacting Dark Matter with Displaced Lepton Jets in p-p collisions  at $\sqrt{s} =$  13 TeV at CMS}}
\author{Maria Jose}
\date{}

\begin{document}
\maketitle
The existence of dark matter is supported by evidence such as rotation curves of galaxies and gravitational lensing. 
Models of self-interacting dark matter (SIDM) could possibly explain the dark matter phenomenon and distribution of dark matter within galaxies. 
In this search for SIDM, the particle dark matter produced at the LHC forms a heavy bound state, which subsequently decays into a pair of boosted, 
long-lived dark photons. The decays of the dark photon could produce clusters of displaced and collimated leptons, which we reconstruct as "displaced lepton jets". 
A search for SIDM has been undertaken at CMS in the 13 TeV p-p collision dataset. 
This talk focuses mainly on measurements of the efficiency of reconstructing the dark photon decays into lepton jets with respect to various signal and detector parameters, 
and the displacement of the lepton jet which would serve as a discriminating parameter between signal and background.


%This talk is based on the ongoing SIDM search performed on the $13$ TeV p-p collisions data taken using the CMS detector. It focuses mainly on understanding the efficiency of reconstructing the dark photon decays into lepton jets with respect to various signal and detector parameters.
%We know the existence of dark matter through astronomical observations such as rotational curves and gravitational lensing. Self-interacting dark matter is a model that could possibly explain the behaviour of dark matter at lower length scales.
%The standard CDM model of dark matter successfully explains the dark matter at large scales but fails to explain issues at lower length scales, including the core-cusp problem. These small-scale issues could be solved by including self-interactions to the dark matter. 
%In my search for a self-interacting dark matter, the dark matter forms a heavy bound state through the mediator, 'a dark photon'. That bound state further decays to a pair of boosted dark photons. Each of those dark photons further decays into a pair of leptons. Since the dark photon is boosted, the leptons we obtain are displaced and collimated. So we reconstruct them into a single object called 'Displaced Lepton Jets'. I present an update on the SIDM search performed on the $13$ TeV p-p collisions data taken using the CMS detector. My work in this search is focused on understanding the efficiency of reconstructing the dark photon decays into lepton jets w.r.t various signal and detector parameters.
%This study is critical as it helps to know how well we can see this signal through the CMS detector.


\end{document}
