\documentclass[12pt]{article}
\usepackage{graphicx, amsmath} % Required for inserting images
\usepackage[left=1in,top=1in,right=1in,bottom=1in]{geometry}
\title{Seminar Abstract}
\author{Maria Jose}
\date{}

\begin{document}
\maketitle
\section*{The Search for Self-Interacting Dark Matter with Displaced Lepton Jets in p-p collisions 
 at $\sqrt{s} =$  13 TeV at CMS and Barrel Timing Layer for MIP Timing Detector}
The existence of dark matter is supported by evidence such as rotation curves of galaxies and gravitational lensing. 
Models of self-interacting dark matter (SIDM) could possibly explain the dark matter phenomenon and distribution of 
dark matter within galaxies. In this search for SIDM, the particle dark matter produced at the LHC forms a heavy 
bound state, which subsequently decays into a pair of boosted, long-lived dark photons. The decays of the dark 
photon could produce clusters of displaced and collimated leptons, which we reconstruct as "displaced lepton jets". A 
search for SIDM has been undertaken at CMS in the 13 TeV p-p collision dataset. The first part of the talk focuses on SIDM 
mainly on lepton jet reconstruction efficiency and the displacement of the lepton jet, a factor which could potentially 
serve as a discriminating parameter between signal and background.


In 2026, The CMS detector will undergo a massive upgrade to tackle the conditions of High Luminosity -LHC (HL-LHC). 
One main challenge it has to tackle is the five times raise in the pileup interactions and it could potentially be 
solved by introducing the MIP Timing Detector (MTD). MTD mainly consists of One Barrel Timing Layer (BTL) and two 
End Cap Timing Layers. At UVA, we are making one-fourth of the total BTL. In second part of the talk I will talk about the progress 
we have made in the production of BTL.


\end{document}
