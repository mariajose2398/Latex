\documentclass{beamer}
\usepackage{geometry, amsfonts, amsmath, tikz, multicol, tikz-imagelabels, 
            multirow, pifont, xcolor, tcolorbox}

\usetheme{Madrid}

\setbeamertemplate{frametitle}[default][center]
\usecolortheme{dove}
\title [BAC Discussion]{ SM Assembly and QA/QC Update}
\author[Maria Jose]{Maria Jose \\ on behalf of the UVA BAC team}
\date{\today}
\institute[UVA]{University of Virginia}
\hypersetup{
    colorlinks=true,
    linkcolor=cyan,
    filecolor=magenta,      
    urlcolor=cyan,
    pdftitle={Overleaf Example},
    pdfpagemode=FullScreen,
    }

\definecolor{uvablue}{RGB}{35,45,75}
\definecolor{uvaorange}{RGB}{229,114,0}
\definecolor{UniBlue}{RGB}{83,121,170}
\definecolor{forestgreen}{RGB}{0,128,0}
\definecolor{bronze}{rgb}{0.8, 0.5, 0.2}
\definecolor{purple}{RGB}{128,0,128}
\definecolor{maroon}{RGB}{184,15,10}
\definecolor{grey}{RGB}{128,128,128}
\definecolor{bondiblue}{rgb}{0.0, 0.58, 0.71}
\definecolor{gold}{RGB}{160,116,10}
\definecolor{peacockblue}{RGB}{0,164,180}

%\setbeamercolor{frame}{bg=black, fg =white}
%\setbeamercolor{frametitle right}{bg=gray!60!white}
%setbeamercolor{palette primary}{bg=black,fg=white}
\setbeamercolor{structure}{fg=uvaorange}
\setbeamercolor*{frametitle}{ fg =peacockblue}
\setbeamercolor*{title}{ fg = peacockblue}
%\setbeamercolor{alerted text}{fg=red!85!black}
\setbeamercolor*{palette primary}{fg =peacockblue}
\setbeamercolor*{palette secondary}{fg =peacockblue}
\setbeamercolor*{palette tertiary}{fg =peacockblue}
\setbeamercolor*{palette quaternary}{fg =peacockblue }
%\setbeamercolor*{background canvas}{bg=uvablue}
%


%\setbeamercolor*{block body}{fg=black,bg=black!10}
%\setbeamercolor*{block title alerted}{,bg=black!15}
%\setbeamercolor*{block title example}{parent(0,164,180)=example text,bg=black!15}

\imagelabelset{
coordinate label font = \sffamily\bfseries\tiny,
coordinate label distance = 1mm,
coordinate label back = white ,
coordinate label text = uvaorange,
annotation font = \normalfont\tiny,
}
%\logo{%
%  
% \includegraphics[width=1cm,height=1cm,keepaspectratio]{../Universitylogos/CMSlogo.png}%
%  \hspace{\dimexpr\paperwidth -2cm}%
% \vspace{\dimexpr\paperwidth }%
% \includegraphics[width=1cm,height=1cm,keepaspectratio]{../UniversityLogos/rotunda.jpg}%
%
%}

\begin{document}

\maketitle
\begin{frame}{Assembly Status}
\begin{itemize}
    \item We have assembled 480 SMs and QA/QCed 432 with the Cesium source.
    \begin{itemize}
        \item We have 402 class A, 9 class B and 21 class C SMs.
    \end{itemize}
    \item Last week we had two 24 SM assembly days.
\end{itemize}
\centering
\includegraphics[width=5cm, height=5cm]{../btl_files/summary_plots/summary_432.png}
\includegraphics[width =5cm, height =5cm]{../btl_files/summary_plots/summary_432_ch.png}\\
\small
Note: The summary plots are before applying the offest correction.
\end{frame}

\begin{frame}{Offset Correction with the Cesium Source}
After we changed DC offeset from 28500 to 22000, we don't see the saturation of the pulses anymore.\\
\centering
Without Offest Correction\\
\includegraphics[width=3cm, height=3cm]{../btl_files/offset_plot/without_offset.png}
\includegraphics[width=3cm, height=3cm]{../btl_files/offset_plot/without_offset_zoom.png}\\
With Offset Correction \\
\includegraphics[width=3cm, height=3cm]{../btl_files/offset_plot/with_offset.png}
\includegraphics[width=3cm, height=3cm]{../btl_files/offset_plot/with_offset_zoom.png}\\
\end{frame}
\begin{frame}
    \frametitle{Before And After Offset Correction}
\centering
Without Offest Correction\\
\includegraphics[width=2.25cm, height=2.5cm]{../btl_files/WithoutOffset/Screenshot 2025-01-27 at 07.00.10.png}
\includegraphics[width=2.25cm, height=2.5cm]{../btl_files/WithoutOffset/Screenshot 2025-01-27 at 07.00.21.png}
\includegraphics[width=2.25cm, height=2.5cm]{../btl_files/WithoutOffset/Screenshot 2025-01-27 at 07.00.32.png}
\includegraphics[width=2.25cm, height=2.5cm]{../btl_files/WithoutOffset/Screenshot 2025-01-27 at 07.00.42.png}
\includegraphics[width=2.25cm, height=2.5cm]{../btl_files/WithoutOffset/Screenshot 2025-01-27 at 07.01.00.png}\\
With Offset Correction \\
\includegraphics[width=2.25cm, height=2.5cm]{../btl_files/WithOffSet/Screenshot 2025-01-27 at 06.56.53.png}
\includegraphics[width=2.25cm, height=2.5cm]{../btl_files/WithOffSet/Screenshot 2025-01-27 at 06.58.07.png}
\includegraphics[width=2.25cm, height=2.5cm]{../btl_files/WithOffSet/Screenshot 2025-01-27 at 07.10.25.png}
\includegraphics[width=2.25cm, height=2.5cm]{../btl_files/WithOffSet/Screenshot 2025-01-27 at 06.57.15.png}
\includegraphics[width=2.25cm, height=2.5cm]{../btl_files/WithOffSet/Screenshot 2025-01-27 at 06.57.33.png}
\\

\end{frame}
\begin{frame}
\frametitle{Taking Data with Sodium Source}
\begin{itemize}
    \item We have tried taking data with the Sodium source and we could not get the fit working for all the channels.
    \item Already made the changes Andrea suggested to do.
    \item We assume that our Sodium source is weak.
\end{itemize}
\centering
\includegraphics[width=5cm, height=5cm]{../btl_files/sodium_fit_fail/Screenshot 2025-01-26 at 10.59.21.png}
\includegraphics[width=5cm, height=5cm]{../btl_files/sodium_fit_fail/Screenshot 2025-01-26 at 10.59.34.png}
\end{frame}
\begin{frame}
    \Huge
    \centering
    THANK YOU
\end{frame}
\end{document}